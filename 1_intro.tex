%%%%%%%%%%%%%%%%%%%%%%%%%%%%%%%%%%%%%%%%%%%%%%%%%%%%%%%%%%%%%%%%%%
\section{Introduction}\label{intro}

 Genetic Algorithm (GA) is a well-known meta-heuristic that is used to seek solutions given a search space. The GA commonly relies on operators such as the crossover operator, the mutation operator and the selection operator. Each of these operators have their own and specific role in the GA. Sawyerr et al. simplify the operators main principles, as follows: The selection operator selects solutions for mating based on the principle of ``survival of the fittest'', while the crossover operator generates new solutions by combining the genetic materials of the selected parents and the mutation operator is an exploratory operator that is applied to the population to sustain diversity.

%TODO: Should I ref the most used?
 One of the most used selection operator is the Tournament operator. It selects individuals based on their fitness value and on a given parameter, named the tournament size. Although, it is widely used selection operator, it has not been proved, mathematically or by experimentation, what is the impact of the selection operator. As a rule of thumb, small values, such as 2 or 3, are widely used. We could trace back to 1991,  and we found out that even from that time is has been a popular choice. However, we must highlight that these values are chosen arbirtrarily and has small scientific background. For instance, in 2009, Nicolau used a more pragmatic approach and set an empirical equation to his binary GA. 
 
 
  we would like to reconsider the idea that exists a trade-off between exploration versus exploitation and the tournament size by verifying, experimentally, the real impact of the tournament size of different domains. 
 
 Consequently, our goal in this paper, is to observe the performance of the tournament selection in a real-valued Genetic Algorithm, named the GA-BBOB. For that we applied the GA-BBOB to 24 noise-free BBOB benchmark functions with 10, 20 and 40 dimensions, for more information about these functions see ~\cite{hansen2010real}. To achieve that, we first explore the number of individuals (k) to be selected by the Tournament operator, with values from 2 to 25. These values are arbitrarily chosen. Then we analyze the results of the GA-BBOB to verify any relationship between all the benchmark functions. Then we propose an alternative, an auto-adaptative k-value ``generator''. This alternative is done with the objective of chosing the ideal value of k and we understand that there is much more that could be done in this field of study.

 From the works showed here, it is possible to understand that there is a preference to small values for the tournament size, as 2 or 3. We could trace back to 1991, and we found out that even from that time is has been a popular choice. However, we must highlight that these values are chosen arbirtrarily and has small scientific background. For instance, in 2009, Nicolau used a more pragmatic approach and set an empirical equation to his binary GA. 
 
 

%TODO:
%change this when the scope is complete
These things are described here and there. we do this and that here and there too. results and other stuff are here and there.