The models that used the Window declustering catalogue have in the
word Window appended at their names or SLC, for the methods that used
the Single Link Cluster.


\section{Experimental Design}
The first experiment was made to compare the all the models proposed
with each other and to discover which method would achieve higher
log-likelihood values. We created some scenarios (space/time regions),
and we applied the methods for for the regions of Kanto, Kansai,
Touhoku and East Japan for a given year (2005-2010) with earthquakes
with depth lesser than 100km. We also used 3 kinds of catalogues with
the minimum magnitude of 3.0: the JMA and the declustered catalogues
form the Window method and the SLC method.

We compared the means of the models log-likelihood values using the
ANOVA test. If a group of variables considered for the ANOVA test
showed no statistically significant difference, we applied the Paired
Student t-test, in the case all groups showed statistically
significant difference, the Tukey HSD methodology analysis was used.

The second experiment was made a to compare how the magnitude of the
earthquakes influence the models generated. We used the same scenarios
from the first experiment. We split the models obtained from these
scenarios into slices composed of earthquakes that have magnitude in a
given magnitude interval. We calculated the log-likelihood of these
slices-models and applied the ANOVA test and the Tukey HSD to compared
them.


\subsection{Details of the Statistical Analysis}\label{anova}
The goal is to discover if there is any variation between the methods
and which are the most influential variables. To achieve that, we used
the ANOVA test, because it indicates that the means of several groups
are equal or not for a given confidence interval. The confidence
interval was set to 95\%, meaning that if the ``p-value'' is smaller
than 0.05 it signifies that there exists a statistical significant
evidence that the variables variance are different from each other.

The hypothesis for this experiment can be generalised as follows:

$$\begin{cases} H_0: \text{The population means are equal.} &\\
H_1: \text{The population means are different.}&\\
\end{cases}$$\\

The Tukey HSD is applied on the results obtained from the ANOVA test
to specify which groups differ, in the case any group has a
``p-value'' lesser than 0.05. Tukey's methodology analysis shows the
means of a case with the means of every other case. Doing so, it
identifies differences between means :

$$\begin{cases}
\mu_a-\mu_b, \text{where $\mu_a$ is the mean of the first group}\\
                \text{$\mu_b$ is the mean of the second group.}
\end{cases}$$

In the case where statistical significant difference exists, we explore this by pairing the measures observations of two groups. That is:

$$\begin{cases}
H_0: \mu = 0, \text{the difference between observations is 0.}&\\
H_1: \mu != 0, \text{difference between observations is not 0.}
\end{cases}$$\\


\subsection{Results from The Mainshock Models Mainshock with Aftershock Models Experiment}\label{resultsBigExp}

An one-way between subjects ANOVA was conducted to compare the effects
of the models, the years and regions on the log-likelihood value. The
models compared are: ReducedGAModel, Emp-ReducedGAModelSLC,
Emp-GAModel, Emp-ReducedGAModel, GAModelWindow, ReducedGAModelWindow,
GAModelSLC, ReducedGAModelSLC, Emp-GAModelWindow,
Emp-ReducedGAModelWindow, GAModel and Emp-GAModelSLC.

Based on the results of the ANOVA test, it is evident that all
variables are significantly different. The results of the experiments
are in the Table~\ref{anovatest1}. All variables had a ``p-value''
lower than 0.05, indicating that they can be considered different.

%TODO: pegar monografia e ver o que eu tinha pensado
\begin{table*}[]
  \centering
  \begin{tabular}{|l|l|l|l|l|l|}
    \hline
	{Variable} & {Degrees of Freedom} & {Sum Sq}    & {Mean Sq}   & {F Value} & {Pr(\textgreater F)} \\
	\hline
	Model    & 15           	  & 149303768  & 9953585   & 63.72    & \textless2e-16     \\
	\hline
	Year     & 5                  & 414016420  & 82803284  & 530.06   & \textless2e-16     \\
	\hline
	Region   & 3                  & 869821655  & 289940552  & 1856.02   & \textless2e-16	\\    
	\hline
  \end{tabular}
  \caption{ANOVA Test Results Values - Mainshock Models Mainshock and Aftershock Models.}
  \label{anovatest1}
\end{table*}

Because we found statistically significant result, we applied a
Post-hoc comparisons using the Tukey HSD analysis methodology on the
ANOVA result. It compared each condition with all others. For example,
it compares the values from the GAModel with the GAModelWindow.

It indicated that the models GAModelSLC, Emp-ReducedGAModelWindow,
GAModelWindow, ReducedGAModelWindow, ReducedGAModelSLC achieve
statistically better or equal results in terms of log-likelihood when
compared with the other models. When they are compared with
themselves, they are statistically equal.

We applied the ANOVA test now only with this 5 models. The models
group have a ``p-value'' of 0.171, indicating that they have similar
log-likelihood values. The results are in the Table~\ref{anovatest2}.
     
\begin{table*}[]
  \centering
  \begin{tabular}{|l|l|l|l|l|l|}
    \hline
	{Variable} & {Degrees of Freedom} & {Sum Sq}    & {Mean Sq}   & {F Value} & {Pr(\textgreater F)} \\
	\hline
	Model    & 4           	  & 884882  & 221220   & 1.604    & 0.171    \\
	\hline
	Year     & 5                 & 150297410  & 30059482  & 217.955   & \textless2e-16     \\
	\hline
	Region   & 3                  & 234225270  & 78075090  & 566.107   & \textless2e-16	\\    
	\hline
  \end{tabular}
  \caption{ANOVA Test Results Values - Emp-ReducedGAModelWindow, GAModelWindow, ReducedGAModelWindow, GAModelSLC, ReducedGAModelSLC.}
  \label{anovatest2}
\end{table*}

Therefore, to confirm that the models are statistically equal, we
conducted the Tukey HSD on this ANOVA result. This time, we found
statistically significant difference only for the year and region
groups. To show that the models results are not statistically
different from each other, we applied a pairing analysis.

From the the pairing analysis, we decided to use the
\textit{ReducedGAModelSLC} as the representative method of this
study. That is because, in all cases when its values were compared, it
showed a better performance in the means of the log-likelihood values
and in only one case the ``p-value'' was higher than 0.05. For the
results, see the Table~\ref{Paired}.

%In this Table, the column labelled $\mu_a - \mu_b$ shows the result
%of paired difference between the models referred in the ``Models
%Compared'' column. The ``p-value'' shows the significance value of
%the paired {\it Student's t-test} for the null hypothesis ``The
%paired difference of the means of the models is equal''.


\begin{table*}[]
  \begin{center}
    \begin{tabular}{|c|c|c|c|}
      \hline
      \multicolumn{1}{|c|}{Region} &
      \multicolumn{1}{|c|}{Models Compared} & \multicolumn{1}{|c|}{Mean of $\mu_a - \mu_b$}&
      \multicolumn{1}{|c|}{p-value} \\
      \hline
      
      Kansai & \textbf{EMP-GAModelWindow} - GAModelWindow &
      38.67553 &  3.304e-05 \\
      
      & \textbf{EMP-GAModelWindow} - ReducedGAModelWindow & 4.272185  & 0.2607\\
      
      & \textbf{EMP-GAModelWindow} - GAModelSLC & 
      112.0424 &  1.122e-05\\
      
      &EMP-GAModelWindow - \underline{\textbf{ReducedGAModelSLC}} &  
      -1.787262 & 0.5673 \\
      
      & GAModelWindow - \textbf{ReducedGAModelWindow} &
      -34.40335 & 0.000963\\
      
      & \textbf{GAModelWindow} - GAModelSLC &
      73.36687  &9.065e-06\\
      
      & GAModelWindow - \underline{\textbf{ReducedGAModelSLC}} &
      -40.46279 & 6.32e-05\\
      
      & \textbf{ReducedGAModelWindow} - GAModelSLC &
      107.7702  & 2.632e-05\\
      
      & ReducedGAModelWindow - \underline{\textbf{ReducedGAModelSLC}} &
      -6.059447 &0.2982\\
      
      & GAModelSLC - \underline{\textbf{ReducedGAModelSLC}} &
      -113.8297 & 1.2e-05\\
      
      \hline
      Touhoku & EMP-\textbf{GAModelWindow} - GAModelWindow &
      3.34556 & 0.546\\
      & \textbf{EMP-GAModelWindow} - ReducedGAModelWindow &
      81.60965 & 5.225e-07\\
      &	\textbf{EMP-GAModelWindow} - GAModelSLC &
      63.02216 &0.01971\\
      & EMP-GAModelWindow - \underline{\textbf{ReducedGAModelSLC}} &
      -62.70586 & 0.007075\\
      & \textbf{GAModelWindow} - ReducedGAModelWindow & 
      78.26409 & 2.938e-05\\
      & \textbf{GAModelWindow} - GAModelSLC &
      59.6766 & 0.04829\\
      & GAModelWindow - \underline{\textbf{ReducedGAModelSLC}} &
      -66.05142 &0.001231\\
      & ReducedGAModelWindow - \textbf{GAModelSLC} &
      -18.58749 & 0.3443\\
      & ReducedGAModelWindow - \underline{\textbf{ReducedGAModelSLC}} &
      -144.3155 & 0.000214\\
      & GAModelSLC - \underline{\textbf{ReducedGAModelSLC}} &
      -125.728 &0.01216\\
      
      
      \hline
      East Japan & \textbf{EMP-GAModelWindow} - GAModelWindow &
      1.872764  & 0.9539\\
      
      & \textbf{EMP-GAModelWindow} - ReducedGAModelWindow &
      194.4944 & 1.834e-06\\
      & \textbf{EMP-GAModelWindow} - GAModelSLC &
      189.1155 & 0.0003456\\
      & EMP-GAModelWindow - \underline{\textbf{ReducedGAModelSLC}} &
      -274.9858 & 4.961e-05\\
      & \textbf{GAModelWindow} - ReducedGAModelWindow &
      192.6217 & 0.003738\\
      & \textbf{GAModelWindow} - GAModelSLC &
      187.2428 &9.495e-06\\
      & GAModelWindow - \underline{\textbf{ReducedGAModelSLC}} &
      -276.8586 & 4.636e-05\\
      & ReducedGAModelWindow - \textbf{GAModelSLC} & 
      -5.378912 & 0.8576\\
      & ReducedGAModelWindow - \underline{\textbf{ReducedGAModelSLC}} &
      -469.4803 & 1.446e-05\\
      & GAModelSLC - \underline{\textbf{ReducedGAModelSLC}} &
      -464.1014  & 2.38e-06\\
      
      
      \hline
      Kanto & EMP-\textbf{GAModelWindow} - GAModelWindow &
      57.95612 & 0.00138\\
      & \textbf{EMP-GAModelWindow} - ReducedGAModelWindow &
      79.60781 & 3.441e-05\\
      & \textbf{EMP-GAModelWindow} - GAModelSLC &
      274.3114  & 5.717e-06\\
      & EMP-GAModelWindow - \underline{\textbf{ReducedGAModelSLC}} & 
      -96.61803  & 6.22e-07\\
      & \textbf{GAModelWindow} - ReducedGAModelWindow & 
      21.65169  & 0.1105\\
      &\textbf{GAModelWindow} - GAModelSLC &
      216.3553  & 2.302e-07\\
      & GAModelWindow - \underline{\textbf{ReducedGAModelSLC}} &
      -154.5741  & 1.741e-05\\
      & \textbf{ReducedGAModelWindow} - GAModelSLC &
      194.7036  & 3.678e-05\\
      & ReducedGAModelWindow -\underline{\textbf{ReducedGAModelSLC}} &
      -176.2258 & 4.337e-06\\
      & GAModelSLC - \underline{\textbf{ReducedGAModelSLC}} &
      -370.9294 &1.942e-06\\
      \hline
    \end{tabular}
  \end{center}
  \caption{Paired Experiment Result. The bold models are the ones with
    better results. The ReducedGAModelSLC always achieved better
    results, therefore it is also underlined.}
  \label{Paired}
\end{table*}



%TODO: add only the real and the best model figures
\subsubsection{The Models Examples And The Real Data}
The Figure~\ref{ReducedGAModelSLC-} shows a model from the
ReducedGAModelSLC method for the year 2005 in East Japan. The next
Figure,~\ref{reduced2005eastjapan} shows a model from the
ReducedGAModel~\ref{reducedGAModel} method for the year 2005 in East
Japan.

All
Figures,~\ref{gamodel2005eastjapan}~\ref{reduced2005eastjapan}~\ref{hybridgamodel2005eastjapan}~\ref{hybridreduced2005eastjapan},
indicate a low earthquake intensity as white while the more intensity
areas, are shown in red. They are, in order, the data visualisation
for the model from: the GAModel, the ReducedGAModel, the Emp-GAModel
and the Emp-ReducedGAModel for East Japan in 2005. The
Figure~\ref{real2005eastjapan} represents the earthquake occurrences
in the same region and year.


%\begin{figure}[H]
%	\centering
%	\includegraphics[scale=0.2]{img/gamodel2005eastjapan.png}
%	\label{gamodel2005eastjapan}
%\end{figure}
%
%
%\begin{figure}[H]
%	\centering
%	\begin{minipage}{0.45\textwidth}
%		\centering
%		\includegraphics[scale=0.2]{img/reduced2005eastjapan.png}
%		\label{reduced2005eastjapan}
%	\end{minipage}
%	\begin{minipage}{0.45\textwidth}
%		\centering
%		\includegraphics[scale=0.2]{img/hybridgamodel2005eastjapan.png}
%		\label{hybridgamodel2005eastjapan}
%	\end{minipage}
%	\caption{The Figure on the left is the ReducedGAModel model for the year of 2005, East Japan, and the one on the right Emp-GAModel model for the year of 2005, East Japan.}
%\end{figure}
%
%\begin{figure}[H]
%	\begin{minipage}{0.45\textwidth}
%		\centering
%		\includegraphics[scale=0.2]{img/real2005eastjapan.png}
%		\caption{Earthquake occurrences in the year of 2005 in East Japan.}
%		\label{real2005eastjapan}
%	\end{minipage}
%	\begin{minipage}{0.45\textwidth}
%		\centering
%		\includegraphics[scale=0.2]{img/hybridreduced2005eastjapan.png}
%		\label{hybridreduced2005eastjapan}
%	\end{minipage}
%	\caption{The Figure on the left is the Emp-ReducedGAModel model for the year of 2005, East Japan, and the one on the right GAModel model for the year of 2005, East Japan, East Japan.}
%\end{figure}




\subsection{Magnitude Experiment}\label{magExp}
The goal is to discover if there is any variation in the methods when
considering the magnitude of the earthquakes in the models. We wanted
to explore the relation between the magnitude of the earthquakes and
how would the models behave on those situations. To achieve that, we
used the ANOVA test The confidence interval was set to 95\%.

%For that, we created magnitude intervals, where each interval is
%named as a slice. A slice is an closed interval of 1.0 degree
%starting from 3.0 degrees of magnitude, see~\ref{catalogs}, and
%ending in 10.0 degrees. For example, $[3.0-4.0]$ or $[7.0-8.0]$ are
%two different slices. For each model, we selected only the
%earthquakes that belong to a slice. Then, we calculate the
%log-likelihood value.
\subsubsection{Magnitude Study}

We compared those split-models against themselves. Based on the
results of this test, it is evident that all variables are still
significantly different. The results of the ANOVA test are in the
Table~\ref{anovatestMag}. For all, as before, we choose the confidence
level to be 95\%.

\begin{table*}[!ht]
  \centering
  \begin{tabular}{|l|l|l|l|l|l|}
    \hline
	{Variable} & {Degrees of Freedom} & {Sum Sq}    & {Mean Sq}   & {F Value} & {Pr(\textgreater F)} \\
	\hline
	Model       & 5            	  & 2.360e+09      & 4.720e+08     & 2828     & \textless2e-16     \\
	\hline
	Year        & 3                  & 4.624e+09   & 1.541e+09    & 9234     & \textless2e-16     \\
	\hline
	Magnitude   & 7                  & 3.726e+09   & 5.322e+08    & 3189     & \textless2e-16	\\    
	\hline
  \end{tabular}
  \caption{ANOVA Test Results Values - Magnitude Study.}
  \label{anovatestMag}
\end{table*}

We found statistically significant result and, as before, we applied
the Tukey HSD test. It indicated that the interval $[3.0-4.0]$ always
performed, in terms of log-likelihood values, worse than all other
intervals. this phenomenon also happens in the interval $[4.0-5.0]$,
though in this case, the difference is not as big as the last one. The
other intervals show no significant difference.

From the results found, we decided to chose only earthquakes with
magnitude higher than 4.0 as our threshold value.




%TODO: add future work with Gabriels
%TODO: this section should be a revision of all
