\section{Conclusion}
\label{sec:conclusion}


We proposed an experimental analysis of the impact of the tournament size on the BBOB benchmark functions. We verified that there is little mathematical or experimental study that backgrounds any possible choice of value for the tournament size. Usually, the tournament size is chosen to be 2 or 3, based on ``previous results''.
 
 
We projected two experiments. We analyzed a group of tournament size values and we also explored the changing of the tournament size value trough the execution of the GA. No significant result was discovered.

We are aware that our work is limited to few tournament size values. Also, we understand that the other parameters may have some influence in the results. We know that there is much more that could be done in this field of study, therefore we propose that more investigations need to be done, in order to truly understand the role of the tournament size value related to the quality of final results. For that, we suggest that another kind of adaptative algorithm should be explored. Also, we would like to explore if our results were mainly related with the GA combination of parameters (crossover operator, mutation operator, selection operator, etc).

