Earthquake Risk Models describe the risk of occurrence of seismic
events on an area based on information such as past earthquakes
in nearby regions and the seismic properties of the area under study.
These models can be used to help to better understand earthquakes,
their patterns and their mechanisms. In previous work, we showed that Genetic Algorithms can be used to
generate earthquake risk models.
%Based on the results of that work, we aimed to explore what are the best configurations for the GAModel.
 In this study we want to analyze any similarities between generating earthquake risk models and the between exploring the search space of BBOB benchmark functions both with Genetic Algorithms. This is done by examining the results diferent values for the selection operator through simulations using the catalog of Japanese earthquakes and the 24 BBOB noise free benchmarks functions with 40 dimensions and comparing their behavior. The results are XXX (here is where i should state that the tournsize ~= 2).