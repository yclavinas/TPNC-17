
% TODO: Change the tone of the introduction, talk about exploring the
% TODO: proposed changes

% TODO: Do not add captions in images (as opposed to axis
% TODO: names). Captions are better added in the latex instead.

\section{Experiment Design}
\label{sec:experiment}

%In this paper we propose three improvements for the GAModel algorithm
%that generates earthquake risk models using GA: A reduced
%representation that limits the search space of the algorithm, a hybrid
%model generation that uses domain knowledge, and the pre-processing of
%the data using Spectral Clustering.
%
%We are interested in determine what effect these improvements have on
%the generation of earthquake risk models. To achieve this, we execute
%a series of simulation experiments. In these experiments, we generate
%earthquake models for each combination of the above modifications,
%using historical earthquake catalog data from the Japanese
%archipelago.

\subsection{Experiment Design}

% Comparison on only East Japan-2006

%% CHANGE HERE TO ADD FOUR REGIONS
%Our experiment has a factorial design with three factors: Using the
%reduced representation, Using the hybrid model, and clustering the
%data set. For each combinations of these factors, we generate 10 models
%for two target regions and 6 five-year periods, for a total of 120
%models per combination.

%% NOT SURE IF WE KEEP THE PER-AREA ANALYSIS OR NOT
%We use ANOVA to test whether any of the combinations shows a
%significant deviation in terms of model accuracy, represented as the
%log-likelihood between the model and the catalog data. If this is
%indicated, we compare each combination against the original algorithm
%using Dunnet's Confidence Interval. In each of these tests, we set
%$\alpha = 0.05$.

%%%%%%%%%%%%%%%%%%%%%%%%%%%%%%%%%%%%%%%%%%%%%%%%%%%%%%%%%%%%%%%%%%%%%%%5
\subsection{Data Sets - GAModel}

%%% Four Regions (Kansai, Touhoku, EastJapan, Kanto)
%%% TWO Regions: EastJapan and Kanto
%%% Six time periods (2005,2006,2007,2008,2009,2010)
%%% Depth cut-offs: 100
%%% Magnitude cutoff: 3

We use the earthquake catalog made available by the \emph{Japan
  Meteorological Agency} (JMA) webpage. From this catalog, we use the
following earthquake data: time of occurrence, latitude,
longitude and  we focus on one area for our study, the ``East Japan'' area.

%\emph{Kanto} is the region around metropolitan Tokyo. In this work we
%define it as the area within latitude 34.8N to 37N and longitude
%138.8W to 141W. It is divided into 2025 bins of approximately
%25km$^2$.

\emph{East Japan} region covers the east coast of Japan, including a
large ocean area. In this work we define it as the area within
latitude 37N to 41N, and 140W to 144W. It is divided into 1600 bins of
approximately 100km$^2$.

% Kanto -> not in the oceanic plaque
% East Japan -> oceanic plaque

%\textbf{Kansai} Kansai is the region that includes Kyoto, Osaka and
%many others historical cities. In this area, rather than Kanto area,
%there is a small seismic activity. Its coordinates are 34 North,
%134.5 West, with 1600 bins. Each bin covers an area of approximately
%25km$^2$.

%\textbf{Touhoku} Touhoku is the region in the North of the main
%Japanese island. It has some clusters of seismic activities during
%the years we studied. Its coordinates are 37.8 North, 139.8 West,
%with 800 bins. Each bin covers an area of approximately 100km$^2$.

%% Time filtering and time slice
%We use earthquakes from the JMA catalog that happen between 2000 and
%2010~\footnote{We deliberately avoid using the 2011 catalog, as the
%  occurrence of the Great East Japan earthquake caused an anomalous
%  number of aftershocks, more than the past five years added
%  together. We are interested in adding this event to our analyses in
%  the future}. This is divided in 11 five-year overlapping periods
%(2000--2005, 2001--2006, and so on), which are identified by the last
%year in the period.
%
%% Depth and Magnitude Filtering
%For each period, we filter earthquakes from the catalog using the
%following rules: 
%\begin{itemize}
%\item Depth of the earthquake must be under below 100km, as shallow
%  earthquakes are considered to be more independent and easier to
%  analyse~\cite{yamanaka1990scaling}.
%\item Magnitude of the earthquake must be above 3. While the catalog
%  lists some earthquakes with magnitude under 3, often such
%  earthquakes escape detection, and thus introduce incompleteness to
%  the catalog.
%\end{itemize}
%
