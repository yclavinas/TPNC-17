
% TODO: Change the tone of the introduction, talk about exploring the
% TODO: proposed changes

% TODO: Do not add captions in images (as opposed to axis
% TODO: names). Captions are better added in the latex instead.

\section{Experiment Design}
\label{sec:experiment}

%In this paper we propose three improvements for the GAModel algorithm
%that generates earthquake risk models using GA: A reduced
%representation that limits the search space of the algorithm, a hybrid
%model generation that uses domain knowledge, and the pre-processing of
%the data using Spectral Clustering.
%
%We are interested in determine what effect these improvements have on
%the generation of earthquake risk models. To achieve this, we execute
%a series of simulation experiments. In these experiments, we generate
%earthquake models for each combination of the above modifications,
%using historical earthquake catalog data from the Japanese
%archipelago.

\subsection{Experiment Design}

% Comparison on only East Japan-2006

%% CHANGE HERE TO ADD FOUR REGIONS
%Our experiment has a factorial design with three factors: Using the
%reduced representation, Using the hybrid model, and clustering the
%data set. For each combinations of these factors, we generate 10 models
%for two target regions and 6 five-year periods, for a total of 120
%models per combination.

%% NOT SURE IF WE KEEP THE PER-AREA ANALYSIS OR NOT
%We use ANOVA to test whether any of the combinations shows a
%significant deviation in terms of model accuracy, represented as the
%log-likelihood between the model and the catalog data. If this is
%indicated, we compare each combination against the original algorithm
%using Dunnet's Confidence Interval. In each of these tests, we set
%$\alpha = 0.05$.

%%%%%%%%%%%%%%%%%%%%%%%%%%%%%%%%%%%%%%%%%%%%%%%%%%%%%%%%%%%%%%%%%%%%%%%5
